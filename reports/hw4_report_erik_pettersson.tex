\documentclass[a4paper]{scrartcl}
\usepackage[utf8]{inputenc}
\usepackage[english]{babel}
\usepackage{graphicx}
\usepackage{lastpage}
\usepackage{pgf}
\usepackage{hyperref}
\usepackage{wrapfig}
\usepackage{fancyvrb}
\usepackage{fancyhdr}
\usepackage[official]{eurosym}
\usepackage{listings}
\usepackage{lmodern}  % for bold teletype font
\usepackage{amsmath}  % for \hookrightarrow
\usepackage{xcolor}
\pagestyle{fancy}

\definecolor{mygreen}{rgb}{0,0.6,0}
\definecolor{myblue}{rgb}{0,0,0.6}
\definecolor{mygray}{rgb}{0.5,0.5,0.5}
\definecolor{mymauve}{rgb}{0.58,0,0.82}

\lstset{ 
  basicstyle=\ttfamily,
  columns=fullflexible,
  postbreak=\mbox{\textcolor{red}{$\hookrightarrow$}\space},
%  backgroundcolor=\color{white},   % choose the background color; you must add \usepackage{color} or \usepackage{xcolor}; should come as last argument
  breakatwhitespace=true,         % sets if automatic breaks should only happen at whitespace
  breaklines=true,                 % sets automatic line breaking
  captionpos=b,                    % sets the caption-position to bottom
  commentstyle=\color{mygreen},    % comment style
  frame=single,	                   % adds a frame around the code
  keepspaces=true,                 % keeps spaces in text, useful for keeping indentation of code (possibly needs columns=flexible)
  keywordstyle=\color{blue},       % keyword style
  language=Java,                 % the language of the code
  numbers=left,                    % where to put the line-numbers; possible values are (none, left, right)
  numbersep=5pt,                   % how far the line-numbers are from the code
  numberstyle=\tiny\color{mygray}, % the style that is used for the line-numbers
  rulecolor=\color{black},         % if not set, the frame-color may be changed on line-breaks within not-black text (e.g. comments (green here))
  showspaces=false,                % show spaces everywhere adding particular underscores; it overrides 'showstringspaces'
  showstringspaces=false,          % underline spaces within strings only
  showtabs=false,                  % show tabs within strings adding particular underscores
  stepnumber=1,                    % the step between two line-numbers. If it's 1, each line will be numbered
  stringstyle=\color{mymauve},     % string literal style
  tabsize=2,	                   % sets default tabsize to 2 spaces
  title=\lstname                   % show the filename of files included with \lstinputlisting; also try caption instead of title
}

% Create header and footer
\headheight 27pt
\pagestyle{fancyplain}
\lhead{\footnotesize{Network Programming, ID1212}}
\chead{}
\rhead{\footnotesize{Homework 4 - Frameworks and Web-based Applications}}
\lfoot{}
\cfoot{\thepage\ (\pageref{LastPage})}
\rfoot{}

% Create title page
\title{Homework 4 - Frameworks and Web-based Applications}
\subtitle{Network Programming, ID1212}
\author{Erik Pettersson - erpette@kth.se}
\date{2018-12-14}

\begin{document}

\maketitle

\section{Introduction}

Task:
\begin{itemize}
	\item{Develop a three-tier web-based application for online currency conversion.}
\end{itemize}

This is done by creating a client-server program which uses the frameworks Spring and Thymeleaf.

Requirements of the solution:
\begin{itemize}
	\item{The application shall have a layered architecture.}
	\item{The \texttt{Converter} must be able to convert between $\geq$ 4 currencies.}
	\item{The \texttt{Client} must be a web browser.}
	\item{Frameworks must be used for all layers. E.g. \texttt{Thymeleaf} for the view, \texttt{Spring} for controllers, and \texttt{Spring/JPA} for the model and integration.}
	\item{The \texttt{Server} must handle transactions, e.g. by using the \texttt{@Transactional} annotation from \texttt{Spring}.}
	\item{The conversion rates must be stored in a database.}
	\item{The user interface must be informative.}
\end{itemize}

\newpage

\section{Literature Study}

I watched the lecture videos for the module and would have read the appropriate chapters in the course book\footnote{Java Network Programming, 4th Edition, Developing Networked Applications, by Elliotte Rusty Harold, O'Reilly \& Ass., Inc., 2013} had it contained any information about Spring and Thymeleaf.\\
I have instead to some degree referred to Oracle's Java Tutorials and Java Documentation pages, but mainly I have looked at Spring's own documentation and guides\footnote{\url{https://spring.io/guides}} along with Baeldung's guides\footnote{\url{https://www.baeldung.com/start-here}}, Thymeleaf's documentation\footnote{\url{https://www.thymeleaf.org/doc/tutorials/2.1/thymeleafspring.html}}, and a hefty dose of Stackoverflow\footnote{\url{https://stackoverflow.com/}} questions.

\section{Method}

I have worked alone but have discussed my work with the assignment with both other students that are currently taking the course and friends of mine that have graduated and now work as IT consultants.\\
I have used IntelliJ IDEA.\\
I have again worked in a compile-to-see-if-it-works way to ensure that the program works and follows the requirements.


\section{Result}

The code can be found at\\
\url{https://github.com/trubb/id1212/tree/master/4_webapp}

\begin{itemize}
	\item{
		Layered architecture\\
		My solution uses a \texttt{Web Browser} as the client for this program. The \texttt{Web Browser} handles everything from connecting to and communicating with the server to displaying the result.\\
		The server uses three layers: Controller, Integration, and Model.\\
		The Controller contains the \texttt{Converter} class, which uses the \texttt{@Controller} mapping in Spring to designate itself as the component that deals with HTTP requests. The \texttt{Converter} is also the component that deals with the calculation of converting between currencies.\\
		The Integration layer houses a Data Access Object (DAO) for the currencies in order to enable actions on the database.\\
		The Model layer contains the classes \texttt{Currency} and \texttt{Conversion} which define the Currency object for the database, and a helper class for performing conversions respectively. In the \texttt{Currency} class we define what columns shall be present in the \texttt{Currency} table in the database and provide getters and setters for acting upon currency objects.
	}
	\item{
		Can convert between at least 4 different currencies\\
		Yes, it is possible to convert between an arbitrary number of currencies as long as a value and a unique code is provided.\\
		Currently the following currencies are represented in the program:\\
		\texttt{
			AED, AFN, ALL, AMD, ANG, AOA, ARS, AUD, AWG, AZN, BAM, BBD, BDT, BGN, BHD, BIF, BMD, BND, BOB, BRL, BSD, BTC, BTN, BWP, BYN, BYR, BZD, CAD, CDF, CHF, CLF, CLP, CNY, COP, CRC, CUC, CUP, CVE, CZK, DJF, DKK, DOP, DZD, EGP, ERN, ETB, EUR, FJD, FKP, GBP, GEL, GGP, GHS, GIP, GMD, GNF, GTQ, GYD, HKD, HNL, HRK, HTG, HUF, IDR, ILS, IMP, INR, IQD, IRR, ISK, JEP, JMD, JOD, JPY, KES, KGS, KHR, KMF, KPW, KRW, KWD, KYD, KZT, LAK, LBP, LKR, LRD, LSL, LTL, LVL, LYD, MAD, MDL, MGA, MKD, MMK, MNT, MOP, MRO, MUR, MVR, MWK, MXN, MYR, MZN, NAD, NGN, NIO, NOK, NPR, NZD, OMR, PAB, PEN, PGK, PHP, PKR, PLN, PYG, QAR, RON, RSD, RUB, RWF, SAR, SBD, SCR, SDG, SEK, SGD, SHP, SLL, SOS, SRD, STD, SVC, SYP, SZL, THB, TJS, TMT, TND, TOP, TRY, TTD, TWD, TZS, UAH, UGX, USD, UYU, UZS, VEF, VND, VUV, WST, XAF, XAG, XAU, XCD, XDR, XOF, XPF, YER, ZAR, ZMK, ZMW, ZWL
		}
	}
	\item{
		Client is a web browser\\
		Yes, it is. I have used \texttt{Firefox} to view the page that is created by the program which, in turn uses \texttt{Thymeleaf} for page creation.
	}
	\item{
		Frameworks all the way down\\
		I use \texttt{Spring} for the controller, \texttt{Spring/JPA} for model and integration, and \texttt{Thymeleaf} for the view.
	}
	\item{
		Server handles transactions\\
		When the client sends a \texttt{POST} request by clicking the "\texttt{calculate}" button the server will try to calculate the result.
	}
	\item{
		Conversion rates stored in a database\\
		I once more use Docker to run a MySQL server as the one-line setup is far superior to any other solution.\\
		I have used \texttt{Euro (\euro{}, EUR)} as the base for all rates. Rate data has been retrieved from \texttt{Fixer} \footnote{\url{https://fixer.io/}} and is put into the database in the \texttt{Application} class.\\
		This could easily be extended to use Fixer's API to download the data directly as a JSON object and put its contents into the database.
	}
	\item{
		Informative user interface\\
		The client is presented with a form containing:\\
		Two dropdowns menus: one for selecting the currency to convert from, and one for selecting the currency to convert to.\\
		A field for inputting the amount to be converted, and a field that will display the sum of the conversion.
	}
\end{itemize}

\subsection{Some code}
\subsubsection{Putting currency data into the database}
Currencies are put into a \texttt{HashMap} \texttt{currencies} with the currency code as the key and the exchange rate as the value.
\lstinputlisting[language=Java, firstline=45, lastline=56]{../4_webapp/src/main/java/id1212/hw4/Application.java}
The currencies are then put into the database as currency objects.
\lstinputlisting[language=Java, firstline=222, lastline=233]{../4_webapp/src/main/java/id1212/hw4/Application.java}

\subsubsection{Serving a HTML page from a Controller}
The \texttt{@Controller}-labeled \texttt{Converter} handles HTTP requests, when first accessed at (in my case) \texttt{localhost:8080/} the HTML file \texttt{converter.html} is served to the browser.
\lstinputlisting[language=Java, firstline=33, lastline=45]{../4_webapp/src/main/java/id1212/hw4/controllers/Converter.java}
When the user has selected currencies to convert between, filled in an amount to convert, and pressed the button labeled \texttt{"Convert"} they are served with a new \texttt{converter.html} page, but this time with a \texttt{Conversion} object that contains their selected currencies along with the amount and the result of the conversion.
\lstinputlisting[language=Java, firstline=47, lastline=67]{../4_webapp/src/main/java/id1212/hw4/controllers/Converter.java}

\subsubsection{Converting between any two currencies}
As we use \texttt{Euro} as the base for all currencies, we can convert between any two currencies by using the following formula: $result = amountFROM / rateFROM * rateTO$.\\
We get the rates by selecting the two currencies from the database and then using the currency objects' \texttt{getRate()} methods.
\lstinputlisting[language=Java, firstline=79, lastline=89]{../4_webapp/src/main/java/id1212/hw4/controllers/Converter.java}


\section{Discussion}

\begin{itemize}
	\item{
		Layered architecture\\
		My implementation has a layered architecture where each discrete layer has a specific role.
	}
	\item{
		Can convert between at least 4 different currencies\\
		Yes, it is possible to convert between an arbitrary number of currencies as long as a value and a unique code is provided. The currencies and their rates are defined in the file \texttt{Application.java}.
	}
	\item{
		Client is a web browser\\
		Yes, it is. I have used \texttt{Firefox} to view the page when working on the assignment. The program uses \texttt{Thymeleaf} for page creation.
	}
	\item{
		Frameworks all the way down\\
		I use \texttt{Spring} for the controller, \texttt{Spring/JPA} for model and integration, and \texttt{Thymeleaf} for the view.
	}
	\item{
		Server handles transactions\\
		The client only displays the HTML pages that it is provided by the server.
		I'm a bit stumped about how container-managed transactions in Spring are done, what is certain is that the server performs the conversions based on client input though, meaning that the client only displays the served pages.
	}
	\item{
		Conversion rates stored in a database\\
		For some reason this felt far easier than in the previous assignment, \texttt{Spring} handled most if not all of the configuration automagically, leaving me to focus on other things.
	}
	\item{
		Informative user interface\\
		The client is presented with a form containing the fields and menus. To build something that presented all information in a ordered manner felt much easier to do in HTML via \texttt{Thymeleaf} compared to e.g. the socket assignments.
	}
\end{itemize}

For this homework I again worked alone, but have discussed the task at hand to a large degree with both fellow students, and people working as IT consultants. 

This assignment seemed complex at first, but as a comparatively extremely large part of the system is obfuscated through the use of frameworks things quickly calmed down a fair bit. The fact that there were lots of documentation for starting new projects with Spring and Thymeleaf meant that it was very easy to do so, especially when compared to the previous assignment.\\
Spring's and Thymeleaf's vast and detailed documentation also helped a lot.


\section{Comments About the Course}

I have spent about the same amount of time on this assignment as I did on assignment 3. This time, however, I have felt far less stressed out about making the deadline, thanks to the longer period.

I am a bit worried about the overlap between assignment 5 and the project. I will probably end up doing the other task from assignment 1 due to this.

Having two weeks allocated for the assignment meant that I was able to feel confident in spending three days just learning about the frameworks and how to use them. This assignment, or the frameworks rather, had way more documentation available online, despite there being zero information in the coursebook.

\end{document}
