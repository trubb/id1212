\documentclass[a4paper]{scrartcl}
\usepackage[utf8]{inputenc}
\usepackage[english]{babel}
\usepackage{graphicx}
\usepackage{lastpage}
\usepackage{pgf}
\usepackage{hyperref}
\usepackage{wrapfig}
\usepackage{fancyvrb}
\usepackage{fancyhdr}
\usepackage{listings}
\usepackage{lmodern}  % for bold teletype font
\usepackage{amsmath}  % for \hookrightarrow
\usepackage{xcolor}
\pagestyle{fancy}

\definecolor{mygreen}{rgb}{0,0.6,0}
\definecolor{myblue}{rgb}{0,0,0.6}
\definecolor{mygray}{rgb}{0.5,0.5,0.5}
\definecolor{mymauve}{rgb}{0.58,0,0.82}

\lstset{ 
  basicstyle=\ttfamily,
  columns=fullflexible,
  postbreak=\mbox{\textcolor{red}{$\hookrightarrow$}\space},
%  backgroundcolor=\color{white},   % choose the background color; you must add \usepackage{color} or \usepackage{xcolor}; should come as last argument
  breakatwhitespace=true,         % sets if automatic breaks should only happen at whitespace
  breaklines=true,                 % sets automatic line breaking
  captionpos=b,                    % sets the caption-position to bottom
  commentstyle=\color{mygreen},    % comment style
  frame=single,	                   % adds a frame around the code
  keepspaces=true,                 % keeps spaces in text, useful for keeping indentation of code (possibly needs columns=flexible)
  keywordstyle=\color{blue},       % keyword style
  language=Java,                 % the language of the code
  numbers=left,                    % where to put the line-numbers; possible values are (none, left, right)
  numbersep=5pt,                   % how far the line-numbers are from the code
  numberstyle=\tiny\color{mygray}, % the style that is used for the line-numbers
  rulecolor=\color{black},         % if not set, the frame-color may be changed on line-breaks within not-black text (e.g. comments (green here))
  showspaces=false,                % show spaces everywhere adding particular underscores; it overrides 'showstringspaces'
  showstringspaces=false,          % underline spaces within strings only
  showtabs=false,                  % show tabs within strings adding particular underscores
  stepnumber=1,                    % the step between two line-numbers. If it's 1, each line will be numbered
  stringstyle=\color{mymauve},     % string literal style
  tabsize=2,	                   % sets default tabsize to 2 spaces
  title=\lstname                   % show the filename of files included with \lstinputlisting; also try caption instead of title
}

% Create header and footer
\headheight 27pt
\pagestyle{fancyplain}
\lhead{\footnotesize{Network Programming, ID1212}}
\chead{}
\rhead{\footnotesize{Homework 2 - Nonblocking Sockets}}
\lfoot{}
\cfoot{\thepage\ (\pageref{LastPage})}
\rfoot{}

% Create title page
\title{Homework 2 - Nonblocking Sockets}
\subtitle{Network Programming, ID1212}
\author{Erik Pettersson - erpette@kth.se}
\date{2018-11-20}

\begin{document}

\maketitle

\section{Introduction}

Task
\begin{itemize}
\item{To write a distributed application that uses nonblocking TCP or UDP sockets.}
\item{To use concurrent threads in nodes of said application which in turn improve scalability and performance}
\end{itemize}

This is done by creating a hangman game with support for multiple concurrent clients, but this time using socketChannels and selectors instead of plain sockets.\\

Requirements of the solution:
\begin{itemize}
\item{The application shall have a layered architecture.}
\item{Only non-blocking sockets are used.}
\item{Both the client and server is multithreaded.}
\item{That the program works.}
\end{itemize}

\section{Literature Study}

I watched the lecture videos for the module. I have also read the chapters about nonblocking I/O and threads in the course book\footnote{Java Network Programming, 4th Edition, Developing Networked Applications, by Elliotte Rusty Harold, O'Reilly \& Ass., Inc., 2013}. Lastly I have referred to Oracle's Java Tutorials and Java Documentation pages to learn how specific functions are used.

Through these sources I learned how to use SocketChannels and ServerSocketChannels, Buffers, and Selectors, and have gained a cursory understanding of the ForkJoinPool class and its uses.

\section{Method}

I have worked alone, using IntelliJ IDEA.\\
This time again I have worked in a compile-to-see-if-it-works way to ensure that the program works and follows the requirements.

\section{Result}

The code can be found at\\
\url{https://github.com/trubb/id1212/tree/master/2_nonblocking_sockets/src}

\begin{itemize}
	\item{
		Layered architecture\\
		My solution uses three layers on the client side: a Controller layer, a Network layer, and a View layer.\\
		The controller passes actions between the different layers.\\
		The Network layer handles the connection to and communication with the server.\\
		The View layer handles user input and output of messages from the server.\\
		\\
		For the Server the layers Network and Model are used.\\
		The Network layer houses the Server's main class, which in turn invokes the \texttt{ClientHandler} for communicating with clients that connect. The \texttt{ClientHandler} then deals with the connected client.\\
		In the Model layer the game class resides, like in the previous homework it is responsible only for performing game-related operations, and returning information about game state to the \texttt{ClientHandler} which in turn sends it to the Client.		
	}
	\item{
		Use of only non-blocking sockets\\
		Only the built-in sockets of the classes \texttt{SocketChannel} and \texttt{ServerSocketChannel} are used, and they have been configured to not block by passing \texttt{false} to their \texttt{configureBlocking()} methods.
	}
	\item{
		Multithreaded Client and Server\\
		Each Client runs its \texttt{Terminal} and \texttt{ServerConnection} classes in separate threads. And it is still possible to disconnect from the server while waiting for a response.\\
		For each key that is ready to accept a connection on its channel a \texttt{ClientHandler} is created, which in turn deals with its client.\\
		Reading of messages from clients is handled in the \texttt{ClientHandler} through a \texttt{ForkJoinPool}'s \texttt{commonPool}.
	}
	\item{
		Program works as expected\\
		Below are a couple of images showing the client UI as well as the server's StdOut, where the selected words as well as client guesses are shown.\\
		In the first image a client plays a round. In the second image a client plays a round but decides to quit before receiving an answer from the server. The third image shows the server's terminal window during all of this.
		\begin{center}
			\includegraphics[width=.3\linewidth]{./images/client_googled.png}\quad\includegraphics[width=.3\linewidth]{./images/client_disconnect.png}\quad\includegraphics[width=.3\linewidth]{./images/server_cut.png}
			\footnotesize{
				Figures 1. 2. \& 3.\\Three images showing 1. a client playing a round. 2. a client playing a round but quitting midgame without waiting for a reply from the server. 3. The server's StdOut terminal when these actions were performed.
			}
		\end{center}
	}
\end{itemize}

\subsection{Some code}

The \texttt{initSelector()} method initializes the server's \texttt{Selector} and \texttt{ServerSocketChannel} by opening a selector; opening and configuring the \texttt{ServerSocketChannel} to not block. Binding the channel to the predefined \texttt{PORTNUMBER} and then registering the channel with the selector.
\lstinputlisting[language=Java, firstline=41, lastline=48]{../2_nonblocking_sockets/src/server/net/Server.java}

The \texttt{readMessage()} method of \texttt{ClientHandler} uses the somewhat misleading syntax to read a message from the \texttt{SocketChannel clientChannel} into the \texttt{ByteBuffer clientMessage} - or as you'd normally put it: write a message from the socketchannel into the buffer. If the amount of read bytes are equal to -1 the channel has reached end-of-stream and we don't have anything to do.
If this hasn't happened then we deserialize the message and add its components to the queue for reading. After this we use \texttt{ForkJoinPool.commonPool().execute( this )} to execute \texttt{ClientHandler.run()} which interprets the message and takes an appropriate action. 
\lstinputlisting[language=Java, firstline=73, lastline=80]{../2_nonblocking_sockets/src/server/net/ClientHandler.java}


\section{Discussion}

\begin{itemize}
	\item{
		Layered architecture\\
		My implementation has a layered architecture where each discrete layer has a specific role.	
	}
	\item{
		Use of only non-blocking sockets\\
		I've only used the classes \texttt{ServerSocketChannel} and \texttt{SocketChannel}. And instead relied on their \texttt{.socket()} methods. The socketchannels have been configured to be nonblocking (\texttt{configureBlocking( false )}.	
	}
	\item{
		Multithreaded Client and Server\\
		Both the Client and Server uses multiple threads to run their components. The Server uses	\texttt{ForkJoinPool.commonPool().execute()} to parse messages in the \texttt{ClientHandler} class.
	}
	\item{
		Program works as expected\\
		The clients connect on startup, can start new game rounds, can quit even if they are waiting for a reply from the server, and are shown the current game state after each guess.\\
		The server starts listening on port \texttt{48922} on startup, and deals with each connecting client in an appropriate manner. The server does not misbehave when a client disconnects.
	}
\end{itemize}

I found the syntax for reading and writing with \texttt{Buffer} and \texttt{Channel} classes to be somewhat confusing. The course book was quite helpful here, but overall the backwards syntax really threw me off.
For this homework I again worked alone, but have sought out more help from peers and friends to clarify any questions I have had. 

It felt like this assignment was a fair bit more complex, most likely because of the extra code required to work with channels and buffers compared to ordinary sockets.

\section{Comments About the Course}

I have spent somewhere around 45-55 hours on this assignment.\\
A very large part of that time was spent reading about the new concepts, classes and methods that were introduced.

\end{document}
